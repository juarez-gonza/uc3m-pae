%----------
%   IMPORTANTE
%----------

% Esta plantilla está basada en las recomendaciones de la guía "Trabajo fin de Grado: Escribir el TFG", que encontrarás en http://uc3m.libguides.com/TFG/escribir
% contiene recomendaciones de la Biblioteca basadas principalmente en estilos APA e IEEE, pero debes seguir siempre las orientaciones de tu Tutor de TFG y la normativa de TFG para tu titulación.



% ESTA PLANTILLA ESTÁ BASADA EN EL ESTILO IEEE


%----------
%	CONFIGURACIÓN DEL DOCUMENTO
%----------

\documentclass[12pt]{extreport} % fuente a 12pt

% MÁRGENES: 2,5 cm sup. e inf.; 3 cm izdo. y dcho.
\usepackage[
a4paper,
vmargin=2.5cm,
hmargin=3cm
]{geometry}

% INTERLINEADO: Estrecho (6 ptos./interlineado 1,15) o Moderado (6 ptos./interlineado 1,5)
\renewcommand{\baselinestretch}{1.15}
\parskip=6pt

% DEFINICIÓN DE COLORES para portada y listados de código
\usepackage[table]{xcolor}
\definecolor{azulUC3M}{RGB}{0,0,102}
\definecolor{gray97}{gray}{.97}
\definecolor{gray75}{gray}{.75}
\definecolor{gray45}{gray}{.45}

% Soporte para GENERAR PDF/A --es importante de cara a su inclusión en e-Archivo porque es el formato óptimo de preservación y a la generación de metadatos, tal y como se describe en http://uc3m.libguides.com/ld.php?content_id=31389625.

% En la plantilla incluimos el archivo OUTPUT.XMPDATA. Puedes descargar este archivo e incluir los metadatos que se incorporarán al archivo PDF cuando compiles el archivo memoria.tex. Después vuelve a subirlo a tu proyecto.
\usepackage[a-1b]{pdfx}

% ENLACES
\usepackage{hyperref}
\hypersetup{colorlinks=true,
	linkcolor=black, % enlaces a partes del documento (p.e. índice) en color negro
	urlcolor=blue} % enlaces a recursos fuera del documento en azul

% EXPRESIONES MATEMÁTICAS
\usepackage{amsmath,amssymb,amsfonts,amsthm}

% Codificación caracteres
\usepackage{txfonts}
\usepackage[T1]{fontenc}
\usepackage[utf8]{inputenc}

% Definición idioma español
\usepackage[spanish, es-tabla]{babel}
\usepackage[babel, spanish=spanish]{csquotes}
\AtBeginEnvironment{quote}{\small}

% diseño de PIE DE PÁGINA
\usepackage{fancyhdr}
\pagestyle{fancy}
\fancyhf{}
\renewcommand{\headrulewidth}{0pt}
\rfoot{\thepage}
\fancypagestyle{plain}{\pagestyle{fancy}}

% DISEÑO DE LOS TÍTULOS de las partes del trabajo (capítulos y epígrafes o subcapítulos)
\usepackage{titlesec}
\usepackage{titletoc}
\titleformat{\chapter}[block]
{\large\bfseries\filcenter}
{\thechapter.}
{5pt}
{\MakeUppercase}
{}
\titlespacing{\chapter}{0pt}{0pt}{*3}
\titlecontents{chapter}
[0pt]
{}
{\contentsmargin{0pt}\thecontentslabel.\enspace\uppercase}
{\contentsmargin{0pt}\uppercase}
{\titlerule*[.7pc]{.}\contentspage}

\titleformat{\section}
{\bfseries}
{\thesection.}
{5pt}
{}
\titlecontents{section}
[5pt]
{}
{\contentsmargin{0pt}\thecontentslabel.\enspace}
{\contentsmargin{0pt}}
{\titlerule*[.7pc]{.}\contentspage}

\titleformat{\section}
{\normalsize\bfseries}
{\thesubsection.}
{5pt}
{}
\titlecontents{subsection}
[10pt]
{}
{\contentsmargin{0pt}
	\thecontentslabel.\enspace}
{\contentsmargin{0pt}}
{\titlerule*[.7pc]{.}\contentspage}


% DISEÑO DE TABLAS
\usepackage{multirow} % permite combinar celdas
\usepackage{caption} % para personalizar el título de tablas y figuras
\usepackage{floatrow} % utilizamos este paquete y sus macros \ttabbox y \ffigbox para alinear los nombres de tablas y figuras de acuerdo con el estilo definido.
\usepackage{array} % con este paquete podemos definir en la siguiente línea un nuevo tipo de columna para tablas: ancho personalizado y contenido centrado
\newcolumntype{P}[1]{>{\centering\arraybackslash}p{#1}}
\DeclareCaptionFormat{upper}{#1#2\uppercase{#3}\par}

% Diseño de tabla para ingeniería
\captionsetup*[table]{
	format=upper,
	name=TABLA,
	justification=centering,
	labelsep=period,
	width=.75\linewidth,
	labelfont=small,
	font=small
}

% DISEÑO DE FIGURAS.
\usepackage{graphicx}
\graphicspath{{imagenes/}} % ruta a la carpeta de imágenes

% Diseño de figuras para ingeniería
\captionsetup[figure]{
	format=hang,
	name=Fig.,
	singlelinecheck=off,
	labelsep=period,
	labelfont=small,
	font=small
}

% NOTAS A PIE DE PÁGINA
\usepackage{chngcntr} % para numeración continua de las notas al pie
\counterwithout{footnote}{chapter}

% LISTADOS DE CÓDIGO
% soporte y estilo para listados de código. Más información en https://es.wikibooks.org/wiki/Manual_de_LaTeX/Listados_de_código/Listados_con_listings
\usepackage{listings}

% definimos un estilo de listings
\lstdefinestyle{estilo}{ frame=Ltb,
	framerule=0pt,
	aboveskip=0.5cm,
	framextopmargin=3pt,
	framexbottommargin=3pt,
	framexleftmargin=0.4cm,
	framesep=0pt,
	rulesep=.4pt,
	backgroundcolor=\color{gray97},
	rulesepcolor=\color{black},
	%
	basicstyle=\ttfamily\footnotesize,
	keywordstyle=\bfseries,
	stringstyle=\ttfamily,
	showstringspaces = false,
	commentstyle=\color{gray45},
	%
	numbers=left,
	numbersep=15pt,
	numberstyle=\tiny,
	numberfirstline = false,
	breaklines=true,
	xleftmargin=\parindent
}

\captionsetup*[lstlisting]{font=small, labelsep=period}
% fijamos el estilo a utilizar
\lstset{style=estilo}
\renewcommand{\lstlistingname}{\uppercase{Código}}

\usepackage{nameref}

\usepackage{float}

\usepackage{color}
\definecolor{lightgray}{rgb}{0.95, 0.95, 0.95}
\definecolor{darkgray}{rgb}{0.4, 0.4, 0.4}
%\definecolor{purple}{rgb}{0.65, 0.12, 0.82}
\definecolor{editorGray}{rgb}{0.95, 0.95, 0.95}
\definecolor{editorOcher}{rgb}{1, 0.5, 0} % #FF7F00 -> rgb(239, 169, 0)
\definecolor{editorGreen}{rgb}{0, 0.5, 0} % #007C00 -> rgb(0, 124, 0)
\definecolor{orange}{rgb}{1,0.45,0.13}
\definecolor{olive}{rgb}{0.17,0.59,0.20}
\definecolor{brown}{rgb}{0.69,0.31,0.31}
\definecolor{purple}{rgb}{0.38,0.18,0.81}
\definecolor{lightblue}{rgb}{0.1,0.57,0.7}
\definecolor{lightred}{rgb}{1,0.4,0.5}
\usepackage{upquote}

\title{Prácticas Externas Curriculares}


\author{Gonzalo Matías Juarez Tello}
\date{}

\begin{document}


%----------
%	PORTADA
%----------
\begin{titlepage}
	\begin{sffamily}
	\color{azulUC3M}
	\begin{center}
		\begin{figure}[H] %incluimos el logotipo de la Universidad
		  \makebox[\textwidth][c]{\includegraphics[width=16cm]{media/logo_UC3M.png}}
		\end{figure}
		\vspace{2.5cm}
		{\Huge Prácticas Externas Curriculares}\\
		\vspace*{0.5cm}
	 	\rule{10.5cm}{0.1mm}\\
		\vspace*{1cm}
		\begin{Large}
		  Gonzalo Matías Juarez Tello\\
                  NIA: 100467578\\
		\end{Large}
	\end{center}
	\vfill
	\color{black}
	\end{sffamily}
\end{titlepage}

\setcounter{page}{2}
\tableofcontents
\thispagestyle{fancy}
 % página en blanco o de cortesía
\thispagestyle{empty}
\mbox{}

\chapter*{Datos de las prácticas}

\begin{itemize}
\item Estudiante:
  \begin{itemize}
  \item Nombres y Apellidos: Gonzalo Matías, Juarez Tello.
  \item NIA: 100467578
  \item Dirección de correo: \url{gjuarez@pa.uc3m.es} o \url{100467578@alumnos.uc3m.es}
  \end{itemize}

\item Empresa:
  \begin{itemize}
  \item \emph{TODO}
  \end{itemize}

\item Periodo de realización:
  \begin{itemize}
  \item Días: \emph{TODO}
  \item Horas: \emph{TODO}
  \item Número de créditos: \emph{TODO}
  \end{itemize}

\item Tutor asignado:
  \begin{itemize}
  \item José Daniel García Sánchez.
  \item Dirección de correo: \url{jdgarcia@inf.uc3m.es}
  \item Tel.: \emph{TODO}
  \end{itemize}
\end{itemize}

\chapter{Introducción}

Yo elegí y tuve la oportunidad de hacer las prácticas con el grupo de
investigación ARCOS. Durante mí estadía, me tocó participar en una
colaboración entre la universidad y EMPRESAA.

\paragraph{}
Esta fue mí primera experiencia siendo miembro de grupo de
investigación y participando de una colaboración entre la industria
y la universidad.

\paragraph{}
En este reporte expongo las tareas y los objetivos de los que fui
responsable y pude cumplir. Comenzando por cuestiones más generales,
como la habilidad para trabajar en equipo y llevar a cabo tareas
técnicas. Seguido por un análisis un poco más profundo de cada fase
del proyecto de la que formé parte.

\paragraph{}
El área en el que se desenvuelve mi participación, es en la de
\emph{Herramientas de Desarrolladores} y \emph{Calidad de Código}. El
primero consiste en una serie de programas que facilitan la
interacción de los programadores con su base de código, dan
información en el momento sobre errores o posibles problemas que hayan
en el código mostrado en pantalla y en la base de código en su
totalidad. En cuanto al segundo item, Calidad de Código, es un
elemento muy relacionado al tema de Herramientas de Desarrolladores,
ya que hace uso de estas para verificar automáticamente un conjunto de
reglas establecido. Estas reglas lo que buscan es acentuar propiedades
deseables en el código, que lo vuelven \emph{seguro, fácil de leer, y
mantenible}.


\chapter{Objetivos generales conseguidos por el estudiante}

Mis responsabilidades variaron entre desafíos técnicos y
comunicacionales. Mí experiencia previa trabajando con equipos era
bastante reducida, por lo que tuve que aprender bastante rápido a
mejorar la comunicación para lograr los mejores resultados posibles.

\paragraph{}
La lista de objetivos generales que considero haber cumplido son:
\begin{itemize}
\item \textbf{Mejorar mi capacidad para trabajar en equipo y entre
  equipos}: Como ya mencioné, fue crucial la comunicación en el
  cumplimiento de los objetivos principales de la colaboración.
\item \textbf{Hacer un análisis en profundidad de distintas soluciones
  al problema en cuestión}: Tuve que ayudar en la selección entre un
  número considerable de posibles Herramientas de Desarrolladores de
  las llamadas \emph{linters} (ej.: Clang Tidy, SonarLint,
  cppcheck). Para lo cual fue necesario hacer un análisis sobre
  tiempos de ejecución, facilidad de adopción al IDE, porcentaje de
  reglas de codificación cubiertas por cada regla, etc..
\item \textbf{Aprender a trabajar en proyectos grandes}: No tenía
  mucha expriencia previa trabajando en bases de código con millones
  de líneas de código. Es una experiencia muy distinta saber ubicarse
  dentro de tales proyectos. Como la herramienta seleccionada no
  implementabla algunas de las reglas acordadas con EMPRESAA, tuve que
  implementarlas yo mismo.
\item \textbf{Comunicar resultados al cliente}: A lo largo del
  proyecto, hubieron varias reuniones con personal de EMPRESAA para
  ponerse al día, elaborar una solución juntos, y comunicar los
  resultados finales. Aprendí como ser conciso para poder ser
  entendido por los miembros del equipo con menos conocimiento
  técnico. Para la adopción de la solución, fue crucial que todos
  entendieran su importancia, por qué es necesario, y cómo las
  Herramientas de Desarrolladores mejoran su experiencia como
  desarrollador.
\end{itemize}


\chapter{Fases de trabajo y objetivos conseguidos en c/u de ellas}

A lo largo del proyecto tuve la oportunidad de participar en distintas
fases.  Comenzando por la definición inicial de reglas de
codificación, seguido por la adaptación de las reglas al
\emph{workflow} de los trabajadores y su inclusión en el
\emph{pipeline de Integración Contínua}, y finalizando en el cierre de
la colaboración y la posterior contribución de código desarrollado a
un proyecto open-source.


\paragraph{}
En concreto, las fases en las que estuve involucrado en orden
cronológico son:
\begin{enumerate}
\item Definición de guía de codificación.
\item Estudio de reglas incluidas en clang-tidy.
\item Implantación en pipelines de Integración Contínua.
\item Desarrollo de reglas adicionales en clang-tidy.
\item Cierre de la colaboración con EMPRESAA.
\item Contribución a proyecto open-source LLVM.
\end{enumerate}

\section{Definición de guía de codificación}
\label{fase-1}

Para esta fase fue crucial la interacción con el cliente. A lo largo
de una serie de reuniones acordamos un conjunto de reglas de
codificación realista para EMPRESAA (reglas que tengan en cuenta la
situación de la empresa, el proyecto, y toda la gente involucrada en
él).

\paragraph{}
En mí opinión, lo interesante de esta fase es que \emph{no solo
requirió del conocimiento técnico} necesario para identificar reglas
que fomenten buenas prácticas, \emph{si no que también fue necesario
saber comunicar esos conocimientos al cliente}.

\paragraph{}
Los objetivos de esta fase son:
\begin{itemize}
\item \textbf{Participar en reuniones virtuales con el cliente}
  (EMPRESAA): Esto fue algo completamente nuevo para mí que nunca me
  había visto en la posición de tener que explicar y trabajar junto a
  un cliente para llegar a un resultado satisfactorio para ambos.

\item \textbf{Formular conjuntos de reglas considerando las
  necesidades de la empresa}, los programadores, y los estándares de
  codificación modernos de C++ (AUTOSAR, MISRA, C++ Core Guidelines):
  Previamente ya tenía una noción de los estándares preexistentes de
  programación en C++ y otros lenguajes, pero aquí me vi en la
  posición de tener que poner en práctica todo este conocimiento.

\item \textbf{Selección de herramienta/s de verificación automática de
  reglas}. Fui responsable hacer un estudio inicial de cada una de las
  distintas alternativas: \emph{SonarLint}, \emph{Microsoft Visual
  Studio Core Guidelines Checker}, \emph{Clang Tidy} y \emph{Clang
  Static Analyzer}, entre otros. Junto al cliente, medimos ventajas y
  desventajas de cada herramienta teniendo en cuenta:
  \begin{itemize}
  \item \emph{Tiempo de ejecución}. Necesitamos herramientas rápidas
    para que el programador no pierda su concentración esperando un
    resultado.
  \item \emph{Extensibilidad}. Capacidad para adoptar nuevas reglas.
  \item \emph{Interactividad}. El programador debe obtener resultados
    mientras codifica para estar en un \emph{feedback-loop} constante.
  \end{itemize}
  Finalmente nos decidimos por usar la suite de Clang con Clang Tidy y
  Clang Static Analyzer, por su extensibilidad y amplia cobertura
  (aunque no total) de las reglas acordadas.
\end{itemize}

\section{Estudio de reglas incluidas en Clang Tidy}
\label{fase-2}

Con las herramientas ya seleccionadas en la primera fase
(especificamente Clang Tidy y Clang Static Analyzer), fue necesario
ampliar sobre el estudio inicial.

\paragraph{}
Tuve la responsabilidad llevar a cabo un un análisis donde se
constataron cuáles de las reglas acordadas con EMPRESAA pueden ser
verificadas automáticamente, cuáles pueden ser verificadas por Clang
Tidy, cuáles deberían ser codificadas por nostoros, y cuáles no pueden
ser codificadas porque dependen de algún factor humano (por ejemplo,
relacionadas al criterio y el ``buen gusto'').

\paragraph{}
Una vez más, la comunicación con el cliente fue esencial para
verificar que se estaba transitando por un camino por el que todos
estamos satisfechos.

\paragraph{}
Los objetivos logrados en esta fase son:
\begin{itemize}
\item \textbf{Llevar a cabo un estudio a fondo sobre las
  verificaciones automáticas ofrecidas por Clang Tidy}: Con estos
  datos, podemos responder las siguientes preguntas:
  \begin{itemize}
  \item ¿Cuántas reglas del conjunto de reglas acordadas es cubierto a
    día de hoy por Clang Tidy?
  \item ¿Cuántas reglas son cubiertas por el compilador de C++ (Clang o GCC)?
  \item ¿Cuántas reglas faltantes deberán ser desarrolladas por nosotros?
  \item ¿Cuántas reglas faltantes no pueden ser verificadas
    automáticamente por depender de criterio humano?.
  \end{itemize}

\item \textbf{Comunicar los resultados al cliente}: De nuevo, fue
  necesario comunicar en más de una reunión los resultados obtenidos
  hasta el momento. De esta manera podemos saber si está de acuerdo
  con la cobertura del conjunto de reglas que logra la herramienta y
  los compromisos posteriores, en particular, el desarrollo de reglas
  faltantes.
  \begin{itemize}
  \item Los desarrolladores están de acuerdo con la herramienta
    seleccionada y la cobertura que ofrece.
  \item Se reconoce la necesidad de desarrollar reglas faltantes sobre
    la base de código de Clang Tidy.
  \end{itemize}
\end{itemize}


\section{Implantación en pipelines de Integración Contínua y Workflow del programador}

Como resultado de las interacciones con el cliente en la fase previa,
se decidió proceder con la adopción de Clang Tidy y Clang Static
Analyzer en el pipeline \emph{Integración Contínua (CI)} y en el
\emph{Workflow} del programador.

\paragraph{}
Iniciando por el workflow, en EMPRESAA usa como IDE (Entorno de
Desarrollo Integrado) principal \emph{Microsoft Visual Studio}. Se
enfrentaron varios problemas para integrar Clang Tidy en este IDE.
Los problemas se deben principalmente a la poca documentación del IDE,
a la naturaleza propietaria que ofusca el proceso de configuración, y
por la estructura orgánica del proyecto de EMPRESAA, que dificulta la
automatización de algunos procesos sobre el proyecto.

\paragraph{}
Haciendo uso del controlador de versiones \emph{Git}, se optó por
escribir un script que ejecute Clang Tidy sobre los cambios efectuados
por el programador cada vez que el programador intente hacer un
\emph{commit}. Algunos desarrolladores en EMPRESAA optan por usar
Linux, por lo tanto este script debía funcionar tanto en Linux como en Windows.

\paragraph{}
Para adoptar Clang Tidy y Clang Static Analyzer al pipeline de
Integración Contínua, descubrimos una herramienta llamada
\emph{CodeChecker}. CodeChecker es básicamente una interfaz a las
herramientas de Clang que ofrece estadísticas muy útiles sobre los
resultados obtenidos, las cuales son visibles mediante interfaces Web
interactivas generadas automáticamente. Esta herramienta es muy fácil
de configurar para su adición a la plataforma de integración contínua
de la empresa.

\paragraph{}
Los objetivos de esta fase entonces son:
\begin{itemize}
\item \textbf{Integrar de Clang Tidy en Microsoft Visual Studio}:
  Solución de problemas de MSVC y el proyecto en sí para la adopción
  de Clang Tidy.
\item \textbf{Desarrollar scripts pre-commit para Git en Linux y
  Windows}: Script que ejecuta Clang Tidy antes de cada commit a
  manera de corroborar que cada pieza de código nuevo está adecuado al
  conjunto de reglas acordados con EMPRESAA.
\item \textbf{Integrar Clang Tidy y Clang Static Analyzer en el
  pipeline de CI haciendo uso de CodeChecker}: Configuración de
  CodeChecker para integrar en Jenkins.  CodeChecker es una
  herramienta que facilita estadísticas obtenidas de la ejecución de
  Clang Tidy y Clang Static Analyzer.
\end{itemize}

\section{Desarrollo de reglas adicionales en Clang Tidy}

En la segunda fase (``\nameref{fase-2}'') confirmamos que varias
reglas no eran verificables automáticamente por Clang Tidy. Algunas de
estas reglas no son verificables por ninguna herramienta al depender
de ciertos aspectos del criterio humano (mayormente problemas de
diseño), pero otras sí que son verificables. El problema con estas
últimas es que el código fuente de Clang Tidy aún no cuenta con
código que implemente tal verificación.

\paragraph{}
Objetivos de esta fase:
\begin{itemize}
\item \textbf{Extender el código fuente de Clang Tidy para cubrir las
  reglas faltantes verificables automáticamente}: Para lo que fue
  necesario:
  \begin{itemize}
  \item Configurar las opciones de compilación del proyecto. Clang
    Tidy es parte de un proyecto llamado \emph{LLVM}. LLVM cuenta con
    millones y millones de líneas de código. Si bien no todas las
    líneas de código son necesarias para producir un binario útil para
    nuestros fines, la compilación sigue siendo un proceso que
    requiere de mucho cómputo. Para poder agilizar el flujo de trabajo
    en la implementación, hice uso de un servidor del cual disponemos
    en ARCOS, el cual tiene prestaciones mucho más altas que las que
    una notebook o un computador de escritorio podrían tener.
  \item Familiarizarse con la base de código de Clang Tidy (muy
    grande). Reitero, LLVM tiene millones de líneas de código. Si bien
    no se trabaja en simultáneo sobre todas las líneas de código, la
    familiarización con la manera en que este proyecto se subdivide y
    las interfaces necesarias para poder trabajar en su extensión, no
    es un proceso trivial.
  \item Escribir el código necesario para cubrir las reglas
    faltantes. Una vez ya familiarizado con las partes de LLVM
    relevantes para la implementación de las reglas faltantes, no fue
    muy complicado escribir el código necesario.
  \item Corroborar que el código escrito conforma el las guías de
    estilo y calidad de LLVM. LLVM dispone varias guías para producir
    código ``aceptable'' y ``uniforme'' con el resto de la base de
    código.
  \end{itemize}
\item \textbf{Facilitar a EMPRESAA los parches realizados sobre Clang
  Tidy}: Con EMPRESAA acordamos la versión de Clang Tidy a usar desde
  un principio, por lo que o solo tuve que realizar un script que
  aplicara de manera automática los parches realizados sobre el código
  de Clang Tidy de esa versión en particular.
\end{itemize}

\section{Cierre de la colaboración con EMPRESAA}

En esta fase se comunicaron los resultados de toda la colaboración a
una porción más grande del equipo de EMPRESAA. Me ví en la posición de
explicar todas las decisiones tomadas y acciones realizadas de una
manera concisa y clara.

\paragraph{}
Objetivos de esta fase:
\begin{itemize}
\item \textbf{Resumir todas las decisiones y acciones realizadas a lo
  largo de la colaboración al equipo de EMPRESAA}: Hacer un repaso de
  todas las fases anteriores, desde la selección de herramienta hasta
  el desarrollo del código para las reglas faltantes en Clang Tidy.
\item \textbf{Trabajar en conjunto con empleados de EMPRESAA}: Para
  desarrollar una explicación clara y concisa para el resto de sus
  compañeros de trabajo.
\item \textbf{Preparar una presentación a modo de resumir todo lo
  realizado en la colaboración}: En donde tuve que explicar y resolver
  dudas del equipo de EMPRESAA previo al cierre de la colaboración.
\end{itemize}

\section{Contribución a proyecto open-source LLVM}

Una vez terminada la colaboración con EMPRESAA, mí tutor fue capaz de
conseguir permiso para poder contribuir las reglas escritas para
EMPRESAA a la base de código de Clang Tidy. Esto beneficia a ambos
bandos; ellos consiguen tener mantenimiento por la comunidad código
abierto de LLVM, y nosotros conseguimos mayor exposición.

\paragraph{}
Objetivos de esta fase:
\begin{itemize}
\item \textbf{Remover referencias a la empresa en el código previo a
  su publicación}: A modo de proteger la privacidad de la empresa.
\item \textbf{Buscar un módulo de Clang Tidy apto para añadir las
  reglas escritas}: Clang Tidy se compone de varios módulos según la
  categoría de problemas que soluciona cada regla (y algunos otros
  criterios varios). Es necesario entonces ubicar las reglas a
  ingresar en uno de estos módulos.
\item \textbf{Contribuir el código y pasar el proceso de revisión de
  LLVM}: El código no puede añadirse a LLVM y ya. LLVM, como proyecto
  bien establecido, sigue una metodología de cambios. Se tienen en
  cuenta cuestiones como motivación del cambio, conformación al estilo
  de código de LLVM y calidad del código en general. Solo una vez
  pasado este proceso, la contribución es adoptada por la base de
  código de LLVM.
\end{itemize}

\chapter{Comentarios sobre las fases realizadas y su relación con el
  grado cursado}

A lo largo del grado, obtuve conocimiento que me fue de gran ayuda
para la realización de la práctica. Siento que las asignaturas que
fueron de mayor ayuda para el problema encarado pertenecen tanto a
ámbitos técnicos como a ámbitos bastante relacionados con habilidades
blandas.

\section{Conocimiento Técnico}

\paragraph{}
\textbf{Procesadores del Lenguaje}

Si de código se trata, mí mayor contribución a lo largo de la práctica
se dió en la infraestructura de compiladores de LLVM. En LLVM tuve que
lidiar con análisis sintáctico y semántico, estar familiarizado con
\emph{Lexers}, \emph{Parse Trees} y \emph{Abstract Syntax
Trees}. Estos temas son explorados en detalle en la asignatura
Procesadores del Lenguaje.

\paragraph{}
\textbf{Arquitectura de Computadores}

Todo el proyecto fue desarrollado en C++. Arquitectura de Computadores
es la primera asignatura (y posiblemente única) en el grado en donde
se tiene contacto con este lenguaje de programación.

\paragraph{}
C++ es un lenguaje complejo, no creo que hubiera sido posible haber
realizado las prácticas con la fluidez con la que se hizo si no
hubiera tenido contacto e interés previo por el lenguaje de
programación en cuestión.

\paragraph{}
\textbf{Estructuras de Datos y Algoritmos}

Como bloque principal de la programación, conocer estructuras de datos
y algoritmos fueron esenciales para comprender el funcionamiento de la
base de código de Clang Tidy.

\paragraph{}
\textbf{Estadística}

Muchos de los datos que tuve que comunicar al equipo de EMPRESAA se
dieron en forma de estadísticos, distribuciones, etcétera. Sin previo
conocimiento del área de estadística, no podría haber comunicado a los
miembros del equipo los resultados obtenidos al evaluar distintas
herramientas que fueron consideradas para la verificación automática
de código.

\section{Ingeniería del Software y Habilidades blandas}

\paragraph{}
\textbf{Técnicas de Expresión Oral y Escrita}

Todo progreso necesitó de una reunión con el resto del equipo. Hablar
de manera clara, precisa y concisa fue crucial para poder comunicar
con el equipo sin causar desinterés en el proceso. Las técnicas
aprendidas en esta asignatura me fueron muy útiles para poder
identificar tiempos y formas en las que distintos temas se deben
comunicar, según la generalidad o especificidad de estos.

\paragraph{}
\textbf{Interfaces de Usuario}

El enfoque de las herramientas estudiadas y reglas desarrolladas se
basan en el área de \emph{Herramientas del Desarrollador}. Estas
herramientas buscan mejorar la \emph{Experiencia de Desarrollo}, que
no es nada más que la experiencia de usuario. Las guías e indicaciones
aprendidas en la asignatura de Interfaces de Usuario fueron necesarias
para poder identificar las necesidades de los desarrolladores.

\paragraph{}
\textbf{Ingeniería del Software}

La identificación de requisitos, necesidades de la empresa, y
comunicación con el usuario son el enfoque principal de esta
asignatura y fueron de vital importancia en todas las fases de estas
prácticas.

\chapter{Consideraciones en beneficio de la realización de PAE}

\section{Experiencia personal}

Me siento conforme con el desarrollo de la práctica. Creo que aprendí
bastante sobre la dinámica de un equipo de trabajo y la forma de
trabajar entre varios equipos de trabajo. Me ví en la posición de
explicar resultados a personal sin mucho conocimiento técnico pero,
aún así, esencial para la empresa. Estas situaciones me ayudaron mucho
a mejorar mi perfil en cuanto a ``habilidades blandas''.

\paragraph{}
En cuanto a la parte más técnica del trabajo, me apasionan los
compiladores y disfruté bastante el desafío de usar LLVM para el
desarrollo de reglas que hacen uso de la infraestructura del
compilador.

\paragraph{}
En general, creo que esta experiencia ha sido muy gratificante. He
aprendido bastante sobre software y la gente que lo produce.

\section{Experiencia en ARCOS}

Mí tutor José Daniel junto a todo el grupo de ARCOS me brindaron un
entorno laboral más que satisfactorio. Fui capaz de ser parte de un
entorno agradable de trabajo, con gente competente que trata con temas
técnicamente interesantes y desafiantes.

\chapter{Conclusiones}

Creo que he desarrollado sustancialmente mí conocimiento a lo largo de
estas prácticas. No solo en cuanto a los conocimientos técnicos
propios de la ingeniería informática, si no también sobre las
habilidades blandas, el poder comunicar ideas y resultados. Creo que
estas prácticas me ayudaron a formarme como un aspirante a ingeniero
más completo del que era previo a su realización.

\paragraph{}
La colaboración con EMPRESAA me enseñó bastante sobre la forma de
operar de la industria IT, sus intereses y prioridades. Así como mí
estadía en ARCOS me mostró la forma de operar de un grupo de
investigación en el área de la ingeniería informática.

\paragraph{}
En cierto nivel, pude ver la manera en que las universidades
interactúan con la industria, y cómo ambas partes se benefician de
este intercambio económico y de conocimiento.

\paragraph{}
Agradezco a José Daniel y a todo el grupo de ARCOS por el excelente
entorno laboral y la oportunidad única que me ofrecieron.


\end{document}
